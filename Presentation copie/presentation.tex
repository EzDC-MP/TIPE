%cc/cv de overleaf.com :s
%RAJOUTER DES DIAPO
\documentclass{beamer}
\usetheme{Frankfurt}
\usecolortheme{seahorse}
\title{Super titre de TIPE}
\subtitle{sous-titre}
\institute{numero d'inscription : 41758}
\author{Pedro ALEXANDRINE}
\date{2020-2021}

\usepackage{xurl}
\usepackage{graphicx}
\graphicspath{ {figures/} }
\setbeamertemplate{footline}[frame number]

\begin{document}

\begin{frame}
	\maketitle
\end{frame}

\begin{frame}
	\frametitle{Sommaire}
	\tableofcontents
\end{frame}

\section{Première approche : simple regression}
\subsection{Principe}
\begin{frame}
	\frametitle{Principe de l'apprentissage supervisée}
	Entrainement donnés X et résultats Y fournis
	\begin{figure}[b]
		\centering
		\includegraphics[scale=0.65]{super_schema}
	\end{figure}
\end{frame}

\begin{frame}
	\frametitle{Principe de l'apprentissage supervisée}
	Optimisation d'une fonction d'objectif
	
	Forme de modèle de régression linéaire:
	\[y(w,x)=w_0+W_1x_1+...+w_px_p\]
	Ici $x$: données, $w$: paramètres du modèle
\end{frame}

\begin{frame}
	\frametitle{ElasticNet}
	\begin{figure}[b]
		\includegraphics[scale=0.3]{EN}
		\centering
		\caption{Exemple avec régression linéaire}
	\end{figure}
	Résultats peu intéressant dans notre cas.
\end{frame}

\subsection{Validation croisée et hyperparamètres}
\begin{frame}
	\frametitle{Recherche du meilleur paramètre}
	Principe de la  validation croisée:
	\begin{figure}[b]
		\centering
		\includegraphics[scale=0.27]{gscv}
	\end{figure}
\end{frame}

\subsection{Résultats avec SVR}
\begin{frame}
	\frametitle{SVR, premier résultat}
	Approche à l'aide du modèle SVR. 
	
	Noyau 'rbf' $\rightarrow$ ajustement du paramètre C
	\begin{figure}[tc]
		\includegraphics[scale=0.2]{SVR_premierdecoup}
		\centering
		\caption{Premier résultat avec SVR et découpage inadapté}
	\end{figure}
\end{frame}

\begin{frame}
	\frametitle{Découpage adapté pour la validation croisée}
		Remise en question de la méthode de découpage
		\begin{figure}[h]
			\centering
			\begin{minipage}{0.5\textwidth}
				\includegraphics[scale=0.3]{kfold}
			\end{minipage}
			\centering
			\begin{minipage}{0.5\textwidth}
				\includegraphics[scale=0.3]{tscv}
			\end{minipage}
		\caption{Comparaison des découpages pour la validation croisée}
		\end{figure}
\end{frame}

\begin{frame}
	\frametitle{SVR}
	\begin{figure}[t]
		\centering
		\begin{minipage}{0.4\textwidth}
			\includegraphics[scale=0.25]{SVR_avant_pt_dinflexion}
		\end{minipage}
		\begin{minipage}{0.4\textwidth}
			\includegraphics[scale=0.25]{SVR_apres_pt_dinflexion}
		\end{minipage}
	\caption{à gauche la prediction avant le pt d'inflexion, à droite, après.}
	\end{figure}
	$\Rightarrow$ Le modèle n'arrive pas à "suivre" sans le point d'inflexion.
\end{frame}


\section{Approche multivariées}
\subsection{Multiregresseur : 'RegressorChain'}
\begin{frame}
	\frametitle{RegressorChain SVR}
	Multiregresseur RegressorChain\\ Corrélation : Cas confirmé $\rightarrow$ hospitalisé $\rightarrow$ décès 
	
	\begin{figure}[h]
		\includegraphics[scale=0.45]{mulitregr_epic_fail}%modifier le fichier
		\caption{Résultat avec SVR insatisfaisant}
	\end{figure}
\end{frame}

\begin{frame}
	\frametitle{RegressorChain TheilSenRegressor}
	Changement de régresseur : meilleurs résultats
	\begin{figure}[h]
		\includegraphics[scale=0.45]{TheilSenRegressor multivariable to predict deces}
	\end{figure}
\end{frame}

\subsection{Réseau neuronal}
\begin{frame}
	\frametitle{Principe du réseau neuronal}
	\begin{figure}[t]
		\centering
		\begin{minipage}{0.5\textwidth}
			\includegraphics[scale=0.2]{nn_sk}
		\end{minipage}
	\end{figure}
\end{frame}

\begin{frame}
	\frametitle{Recherche corélations}
	Corélation pour savoir les données pertinentes
	\begin{figure}
		\includegraphics[width=\textwidth]{cor2}
	\end{figure}
	\begin{figure}
		\includegraphics[width=\textwidth]{cor1}
	\end{figure}
	\begin{figure}
		\includegraphics[width=\textwidth]{cor4}
	\end{figure}
\end{frame}

\begin{frame}
	\frametitle{Premier résultat}
	140 000 iteration sans les cas confirmés
	\begin{figure}[h]
		\centering
		\includegraphics[width=0.7\textwidth]{NN mlpregressor 140k iter (updated data)}
	\end{figure}
	Échec du modèle sans les cas confirmés$\rightarrow$ cohérent avec corrélation
\end{frame}

\begin{frame}
	\frametitle{Résultats}
	Réseaux neuronaux : Meilleurs paramètres
	\begin{figure}
		\includegraphics[scale=0.6]{NN_3}
	\end{figure}
\end{frame}

\begin{frame}
	\frametitle{Prédiction durant l'évolution de la situation}
	Divergence après quelques jours
	\begin{figure}[h]
		\centering
		\includegraphics[width=\textwidth]{NN_1}
	\end{figure}
\end{frame}

\begin{frame}
	\frametitle{Augmentation du décalage}
	Résultats insatisfaisant
	\begin{figure}
		\includegraphics[width=0.95\textwidth]{NN_2}
	\end{figure}
\end{frame}

\begin{frame}
	\frametitle{Conclusion}
	Modèle peu fiable hors situation stabilisée ou début d'évolution
\end{frame}



\end{document}
