%cc/cv de overleaf.com :s
\documentclass{beamer}
\usetheme{Frankfurt}
\usecolortheme{seahorse}
\title{Étude sur le propagation du sars-cov1 avec machine learning}
%\subtitle{sous-titre}
\date{2020-2021}
\institute{numero d'inscription : 41758}
\author{Pedro ALEXANDRINE}


\usepackage{xurl}
\usepackage{graphicx}
\usepackage{appendixnumberbeamer}

\usepackage[utf8]{inputenc}
\usepackage[T1]{fontenc}
\usepackage{csquotes}
\graphicspath{ {figures/} }
\setbeamertemplate{footline}[frame number]

\begin{document}

\begin{frame}
	\maketitle
\end{frame}

\begin{frame}
	\frametitle{Sommaire}
	\tableofcontents
\end{frame}

\section{Première approche : simple regression}
\subsection{Principe}
\begin{frame}
	\frametitle{Principe de l'apprentissage supervisée}%	A DÉGAGER
	Entrainement donnés X et résultats Y fournis
	\begin{figure}[b]
		\centering
		\includegraphics[scale=0.65]{super_schema}
	\end{figure}
\end{frame}

\subsection{Validation croisée et hyperparamètres}
\begin{frame}
	\frametitle{Recherche du meilleur paramètre}
	Principe de la  validation croisée:
	\begin{figure}[b]
		\centering
		\includegraphics[scale=0.27]{gscv}
	\end{figure}
\end{frame}

\subsection{Résultats avec SVR}
\begin{frame}
	\frametitle{SVR, premier résultat}
	Approche à l'aide du modèle SVR.
	
	Noyau «\,rbf\,» $\rightarrow$ ajustement du paramètre $C$
	\begin{figure}[tc]
		\includegraphics[scale=0.2]{SVR_premierdecoup}
		\centering
		\caption{Premier résultat avec SVR et découpage inadapté, $C=50k$}
	\end{figure}
\end{frame}

\begin{frame}
	\frametitle{Découpage adapté pour la validation croisée}
		Remise en question de la méthode de découpage
		\begin{figure}[h]
			\centering
			\begin{minipage}{0.5\textwidth}
				\includegraphics[scale=0.3]{kfold}
			\end{minipage}
			\centering
			\begin{minipage}{0.5\textwidth}
				\includegraphics[scale=0.3]{tscv}
			\end{minipage}
		\caption{Comparaison des découpages pour la validation croisée}
		\end{figure}
\end{frame}

\begin{frame}
	\frametitle{SVR}
	g:c=100k
	d,c=1M
	\begin{figure}[t]
		\centering
		\begin{minipage}{0.4\textwidth}
			\includegraphics[scale=0.25]{SVR_avant_pt_dinflexion}
		\end{minipage}
		\begin{minipage}{0.4\textwidth}
			\includegraphics[scale=0.25]{SVR_apres_pt_dinflexion}
		\end{minipage}
	\caption{à gauche la prediction avant le pt d'inflexion, à droite, après.}
	\end{figure}
	$\Rightarrow$ Le modèle n'arrive pas à "suivre" sans le point d'inflexion.
\end{frame}

\begin{frame}%SUPPRIMER CELUI LA?
	\frametitle{RegressorChain SVR}
	Multiregresseur RegressorChain\\ Corrélation : Cas confirmé $\rightarrow$ hospitalisé $\rightarrow$ décès 
	
	\begin{figure}[h]
		\includegraphics[scale=0.43]{mulitregr_epic_fail}
		\caption{Résultat avec SVR insatisfaisant}
	\end{figure}
\end{frame}

\begin{frame}
	\frametitle{RegressorChain TheilSenRegressor}
	Changement de régresseur : meilleurs résultats
	\begin{figure}[h]
		\includegraphics[scale=0.327]{theiltruc 90000 iter tol 0,001}
		\includegraphics[scale=0.327]{theiltruc 9000 iter tol 0,001; 10 jour de plus}
	\end{figure}
\end{frame}

\subsection{Réseau neuronal}
\begin{frame}
	\frametitle{Principe du réseau neuronal}
	\begin{figure}[t]
		\centering
		\begin{minipage}{0.5\textwidth}
			\includegraphics[scale=0.2]{nn_sk}
		\end{minipage}
	\end{figure}
\end{frame}


\section{Approche multivariées}
\subsection{Multiregresseur : 'RegressorChain'}
\begin{frame}
	\frametitle{Recherche corélations}
	Corélation pour savoir les données pertinentes
	\begin{figure}
		\includegraphics[width=\textwidth]{cor2}
	\end{figure}
	\begin{figure}
		\includegraphics[width=\textwidth]{cor1}
	\end{figure}
	\begin{figure}
		\includegraphics[width=\textwidth]{cor4}
	\end{figure}
\end{frame}

\begin{frame}
	\frametitle{Premier résultat}
	DÉCALAGE COURBE
	\begin{figure}[h]
	\end{figure}
\end{frame}

\begin{frame}
	\frametitle{Premier résultat}
	
	\begin{figure}[h]
		\centering
		\includegraphics[width=0.4615\textwidth]{NN mlpregressor 140k iter (updated data)}
		\includegraphics[width=0.46\textwidth]{NN_1}
		\caption{Résultat avec SVR insatisfaisant}
	\end{figure}
	Échec du modèle sans les cas confirmés$\rightarrow$ cohérent avec corrélation
\end{frame}

\begin{frame}
	\frametitle{Résultats}
	Réseaux neuronaux : Meilleurs paramètres
	\begin{figure}
		\includegraphics[scale=0.6]{NN_3}
	\end{figure}
\end{frame}

\begin{frame}
	\frametitle{Prédiction durant l'évolution de la situation}
	Divergence après quelques jours
	\begin{figure}[h]
		\centering
		\includegraphics[width=\textwidth]{NN_1}
	\end{figure}
\end{frame}

\begin{frame}
	\frametitle{Augmentation du décalage}
	Résultats insatisfaisant (aussi avec 14 jour de décalage)
	\begin{figure}
		\includegraphics[width=0.95\textwidth]{NN_2}
	\end{figure}
\end{frame}

\begin{frame}
	\frametitle{Conclusion}
	Modèle peu fiable hors situation stabilisée ou début d'évolution
\end{frame}

\appendix

\begin{frame}
	\frametitle{Augmentation du décalage}
	Résultats insatisfaisant
	\begin{figure}
		\includegraphics[width=0.95\textwidth]{NN_2}
	\end{figure}
\end{frame}


\end{document}
