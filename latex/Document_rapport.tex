\documentclass[a4paper,11pt]{article}

\usepackage[french]{babel}
\usepackage[T1]{fontenc}
\usepackage[utf8]{inputenc}
\usepackage{lmodern}
\usepackage{microtype}

\usepackage{hyperref}
\usepackage[margin=1in]{geometry} %Change la marge
\usepackage{xurl}

\title{TIPE \\ \small{"Enjeux sociétaux : environnement, sécurité, énergie"}}
\author{DE CARVALHO Enzo et ALEXANDRINE Pedro}
\date{}

\begin{document}
\maketitle
\section*{Sujet du TIPE : Modélisation d'un épidémie à l'aide du Machine Learning}
Dans ce sujet de TIPE, nous chercherons à modéliser la progression d'une épidémie à l'aide du machine learning (à préciser au fil du TIPE!!)

\section*{Ressources :}
\begin{enumerate}
\item Tableau de bord en ligne suivant la propagation du virus sars-cov2 par le CSSE :\\
\url{https://gisanddata.maps.arcgis.com/apps/opsdashboard/index.html#/bda7594740fd40299423467b48e9ecf6}

\item Tableau de bord de la WHO :\\
\url{https://covid19.who.int/}

\item Dépot GitHub récoltant les données nationales françaises en un fichier :\\
\url{https://github.com/opencovid19-fr/data}

\item Dépot Github de données épidémiologiques par le MIDAS :\\ 
\url{https://github.com/midas-network/COVID-19} 
\end{enumerate}

\subsection*{Cours sur le Machine Learning :}

\small{il s'agit ici de ressources nous permettant d'acquérir les connaissances nécessaire en machine learning mener à bien notre TIPE. Néanmoins, elles n'ont pas été toutes assimilées à l'écriture de ce document.}\\
\begin{enumerate}
\item Introduction au réseau neuronal :
\url{https://www.youtube.com/playlist?list=PLZHQObOWTQDNU6R1_67000Dx_ZCJB-3pi}

\item Cours sur le Machine Learning par Andrew Ng :\\
\url{https://www.youtube.com/playlist?list=PLZ9qNFMHZ-A4rycgrgOYma6zxF4BZGGPW}

\item Livre sur l'apprentissage statistique :\\
\url{https://web.stanford.edu/~hastie/ElemStatLearn//printings/ESLII_print12_toc.pdf}
\end{enumerate}
\end{document}