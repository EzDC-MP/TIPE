%cc/cv de overleaf.com :s
\documentclass{beamer}
\usetheme{Frankfurt}
\usecolortheme{seahorse}
\title{Super titre de TIPE}
\subtitle{sous-titre}
\author{Alexandrine, Pedro\\\and De Carvalho, Enzo}
\date{2020-2021}

\usepackage{xurl}

\begin{document}
\begin{frame}
	\frametitle{Sommaire}
	\tableofcontents
\end{frame}

\section{Première approche : simple regression}
\subsection{Regression linéaire (ElasticNet)}
\begin{frame}
	\frametitle{Regression linéaire}
	ElasticNet ne fait que des droites, c'est pas intéressant.
$<$image$>$
\end{frame}

\subsection{Optimisation d'hyperparamètres et stratégies}
\begin{frame}
	\frametitle{frametilte}
	là on explique en quoi consiste l'optimisation d'hyperparamètres (crossvalidation)
	Et pourquoi on est passer du découpage de base que propose GridSearchCV (k-fold) au découpage tscv (time-split series)
	\\
	\alert{lien utile : }
	\url{https://scikit-learn.org/stable/auto_examples/model_selection/plot_cv_indices.html########sphx-glr-auto-examples-model-selection-plot-cv-indices-py}
\end{frame}

\subsection{Résultats avec SVR}
\begin{frame}
	\frametitle{titre de la frame}
	images commentées selon ou non on à le pt d'inflexion
	de svr
\end{frame}

\section{Approche multivariées}
\begin{frame}

\end{frame}
	chose à écrire
\end{document}